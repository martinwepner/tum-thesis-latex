\chapter{Navier Stokes Equations [TODO title]}\label{chapter:cudapressuresolver}

The fundamentals of any fluid simulation lay in the Navier Stokes equations[TODO]:\\
\begin{equation} \label{navier-stokes1}
	\frac{\partial \vec{u}}{\partial t} + \vec{u} \cdot \nabla \vec{u} + \frac{1}{\rho}  \nabla p = \vec{F} + \nu \nabla \cdot \nabla \vec{u}
\end{equation}
\begin{equation} \label{navier-stokes2}
	\nabla \cdot u = 0
\end{equation}
The letter $\vec{u}$ is the velocity vector of the fluid.\\
The Greek letter Rho $\rho$ stands for the density of the fluid. Water for example has a density of 1000 $kg/m^3$. \\
The arabic $p$ on the other hand holds the pressure, which is the force per unit the fluid exerts on its surroundings.\\
External forces like gravity and buoyancy are collected in the letter $\vec{F}$.\\
The Greek letter Nu $\nu$ is called kinematic viscosity. The higher the viscosity of a fluid is the higher its inner inertia. For example honey has a higher viscosity than water.\\\\
The first of those equations - the "momentum equation" - very briefly describes how the fluid accelerates due to the forces acting on it. The second is called "incompressibility condition" and means that the volume of the fluid doesn't change.
\par However, real fluids are compressible, so why do we assume them to be incompressible? Liquid fluids are in theory compressible, but it requires a lot of force to do so and it is also only possible until a tiny degree of compression. Gases like air are easier to compress, but only with the help of pumps or extreme situations like blast waves caused by explosions. Those kind of scenarios are called "compressible flow" and are expensive to simulate. Additionally they are only visible on macroscopic level and hence irrelevant for the animation of fluids.
\par Only in those cases where animating viscous fluids like honey is desired, viscosity needs to be considered. For most cases however, it plays a minor role and can be dropped. Those ideal fluids without viscosity are called "inviscid". The Navier Stokes equations can then be simplified to the following, the so called "Euler Equations":
\begin{equation} \label{navier-stokes12}
	\frac{\partial \vec{u}}{\partial t} + \frac{1}{\rho}  \nabla p = \vec{F}
\end{equation}
\begin{equation} \label{navier-stokes22}
	\nabla \cdot u = 0
\end{equation}

\section{The role of the pressure solve in incompressible Navier Stokes equations}
The pressure solve is part of the Navier Stokes equations:
\begin{equation} \label{navier-stokes-pressure}
	\frac{\partial \vec{u}}{\partial t} + \frac{1}{\rho}  \nabla p = 0 \quad s.t. \quad \nabla \cdot \vec{u} = 0
\end{equation}
This means, that the pressure always needs to be so, that it advects the velocity field $\vec{u}$ divergence-free and thus makes the fluid incompressible.  [TODO: Maybe write \textit{what exactly} the pressure is and what it means for the simulation]

\subsection{Boundary Conditions}
If a inviscid fluid is in contact with a wall respective solid, it should move in the tangential direction of the solids normal. To achieve this the velocity components of the solid normal needs to be zero. [TODO: rewrite this]
\begin{equation} \label{navier-stokes12}
	\vec{u} \cdot \hat{n} = 0
\end{equation}
And for moving boundaries the tangential normal component of the velocity matches the normal velocity of the solid:
\begin{equation} \label{navier-stokes12}
	\vec{u} \cdot \hat{n} = \vec{u}_{solid} \cdot \hat{n}
\end{equation}
Compared to water, air is 700 times lighter. For simplicity the pressure $p$ of air regions may be set to zero since it has only very little impact and therefore no significant visual effect.

\subsection{Staggered MAC Grids}
Grid based fluid simulation have been discussed in this work already, but not yet clarified what's exactly meant by "grid". To numerically approximate the Navier Stokes equations, quantities like pressure, velocity and fluid density needs to be computed through space. As the name suggest those quantities are layed out on some kind of regular grid. [Harlow and Welch 1965] have developed the Marker-and-Cell (MAC) grid technique which is still used in many simulators. Instead of tracking all quantities on the same grid, they are staggered on the grid. This avoids biased or null spaced central differences for the pressure gradient. 
\par TODO Figures of grids 

\subsection{Discretization and derivation of the pressure equation}
In this section we will briefly derive a numerical approximation for the pressure equation of the Navier Stokes equations with the help of a MAC-grid for two dimensional simulations. [Bridson] already explained this for both 2D and 3D in detail.\\
Discretizing (3.5) and (3.7) gives the following equations:
\begin{equation} \label{navier-stokes12}
	\vec{u}^{n+1} = \vec{u}^{n} - \Delta t \frac{1}{\rho} \nabla p \quad s.t. \quad \nabla \cdot \vec{u}^{n+1} = 0
\end{equation}
\begin{equation} \label{navier-stokes12}
	\vec{u}^{n+1} \cdot \hat{n} = \vec{u}_{solid} \cdot \hat{n}
\end{equation}
The divergence of a 2 dimensional vector $\vec{u} = (u, v)$ is defined as
\begin{equation} \label{navier-stokes12}
	\nabla \cdot \vec{u} = \frac{\partial u}{\partial x} + \frac{\partial v}{\partial y}
\end{equation}
So for a fluid grid cell $(i, j)$:
\begin{equation} \label{navier-stokes12}
	(\nabla \cdot \vec{u})_{i,j} \approx \frac{u_{i+1/2,j}^{n+1} - u_{i-1/2,j}^{n+1}}{\Delta x} + \frac{v_{i,j+1/2}^{n+1} - v_{i,j-1/2}^{n+1}}{\Delta x} = 0
\end{equation}
Using the central difference the pressure update is:
\begin{equation} \label{navier-stokes12}
	u_{i+1/2,j}^{n+1} = u_{i+1/2,j}^{n} - \Delta t \frac{1}{\rho} \frac{p_{i+1,j} - p_{i,j}}{\Delta x}
\end{equation}
\begin{equation} \label{navier-stokes12}
v_{i+1/2,j}^{n+1} = v_{i,j+1/2}^{n} - \Delta t \frac{1}{\rho} \frac{p_{i,j+1/2} - p_{i,j}}{\Delta x}
\end{equation}
Substituting (3.12) and (3.13) into (3.11) finally gives:
\begin{equation}
	\begin{aligned}
		& \frac{1}{\Delta x}\left[\left(u_{i+1/2,j}^{n} - \Delta t \frac{1}{\rho} \frac{p_{i+1,j} - p_{i,j}}{\Delta x}\right) - \left(u_{i-1/2,j}^{n} - \Delta t \frac{1}{\rho} \frac{p_{i-1,j} - p_{i,j}}{\Delta x}\right) \right. \\
		& + \left. \left(v_{i,j+1/2}^{n} - \Delta t \frac{1}{\rho} \frac{p_{i,j+1} - p_{i,j}}{\Delta x}\right) - \left(v_{i,j-1/2}^{n} - \Delta t \frac{1}{\rho} \frac{p_{i,j-1} - p_{i,j}}{\Delta x} \right) \right] = 0 \\
		& \frac{\Delta t}{\rho} \left( \frac{4p_{i,j} - p_{i+1,j} - p_{i,j+1} - p_{i-1,j} - p_{i,j-1}}{\Delta x^2} \right) = - \left( \frac{u_{i+1/2,j}^{n} - u_{i-1/2,j}^{n}}{\Delta x} + \frac{v_{i,j+1/2}^{n} - v_{i,j-1/2}^{n}}{\Delta x} \right)
\end{aligned}
\end{equation}
Now we have a system of linear equations in the form of $\mathbf{A}p = -\nabla \cdot \vec{u}^n$. $\mathbf{A}$ is symmetric and each row represents one cell in the grid. 


