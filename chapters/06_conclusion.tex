\chapter{Conclusion and Future Work}\label{chapter:conclusion}

In this thesis, I presented an efficient implementation for the pressure solver. I introduced the Tailored CSR format that allows the Laplace Matrix to be created efficiently on the GPU with little memory usage. This allows the matrix multiplication in the cg-method to be calculated faster than conventional methods. The highly optimized cuBLAS library can execute all other mathematical operations efficiently on the GPU. The source of this work is available in the mantaflow git repository including the benchmark scripts.\\\\
I was able to improve the speed by a factor of up to 139 compared to the CPU version and by a factor of up to 72 compared to the $\Phi_{flow}$ TensorFlow GPU implementation. This will help to speed up the training process considerably.  \\\\
Because the main focus of this work was the acceleration of the pressure solver, there is still potential to improve the Laplace matrix generation kernel. A fast method is crucial when objects move through space during the simulation. Furthermore, there is certainly room for further optimizations to the pressure solve kernel. At the moment it scales pretty much with the number of batches. That can be addressed in future work and with the help of CUDA experts, who can utilize the resources of GPUs more effectively than myself. 