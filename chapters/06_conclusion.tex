\chapter{Conclusion and Future Work}\label{chapter:conclusion}

In this thesis, I presented an efficient implementation for the pressure solver. I introduced the Tailored CSR format that allows the Laplace Matrix to be created efficiently on the GPU with little memory usage. This allows the matrix multiplication in the cg-method to be calculated faster than conventional methods. The highly optimized cuBLAS library can execute all other mathematical operations efficiently on the GPU. The source code of this work is available in the mantaflow git repository including the build and benchmark scripts.\\

I was able to improve the speed of the pressure solve compared to the multicore CPU version of $\Phi_{Flow}$ by a factor of up to 99 and by a factor of up to 74 compared to its GPU implementation. Furthermore creating the Laplace Matrix is more than 800 times faster on large enough grid sizes than the comparative solution. This will help to speed up the training process of the neural networks considerably, even with moving objects. \\

Because the main focus of this work was the acceleration of the pressure solver, there is still potential to improve the Laplace Matrix generation kernel. Currently it does not support curved boundaries, which can be addressed in the future.
\par Furthermore, there is certainly room for further optimizations to the pressure solve kernel. At the moment it scales pretty much with the number of batches. That can be addressed in future work and with the help of CUDA experts, who can utilize the resources of GPUs more effectively than myself. 